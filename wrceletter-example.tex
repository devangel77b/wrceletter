\documentclass[12pt,courier]{navyletter}

\author{Dennis Evangelista}
\title{Fictional technical correspondence example}
\navysubj{Fictional technical correspondence example}
\navyfiling{}
\navyserial{}
\date{\today}
\navymarking{for training use only}

\begin{document}
\makedateblock{}

\begin{navyletterheader}
\navyfrom{Commander, Naval Sea Systems Command, Code 08E}
\navyto{General Manager, Bettis Atomic Power Laboratory}
\navyvia{Assistant NAVSEA Technical Representative, Pittsburgh, PA}
\navyskip{}%
\navysubjline{}%
\navyskip{}%
\navyref{refa}{CVN-71 letter Ser 1234, dtd 2 Jan 2016}
\navyref{refb}{NAVSEA 250-1500-1 Nuclear Welding Manual}
\navyref{refc}{Bettis letter Ser 5678, dtd 1 Feb 2016}
%\navyserial{}
%\navyencl{encl1}{blah}
\end{navyletterheader}

\section{Discussion}\label{test} 
One way in which you may encounter technical writing in the Fleet is in the form of writing letters that document technical decisions, establish programs, or issue technical irection to contractors or Fleet personnel.  It is useful to think about what must go in such letters. This is paragraph~\ref{test}.

\section{} 
For a technical issue, you will typically provide numbered paragraphs that identify what is known about the issue. Often these will be divided up, such as describing a Fleet or contractor request; then a separate paragraph discussing technical bases for a decision. 

\section{CVN-71 Request}  
For example, \navyreference{refa}, a manual change request from operators, recommends changing the word "happy" in \navyreference{refb}, operating procedure (OP) 21A (Starting up a Welding Machine), to "glad".  \Navyreference{refc}\ recommends making this editorial change in the next regular revision of reference (b). 

\section{NAVSEA Action}  Technical letters generally follow a format of fact - discussion - action.  In this sort of fake example, the request would be denied; changing the word happy to glad is not a substantive technical change that has any real impact on the system and is not needed.  The revisions to reference (b) requested by reference (a) are disapproved. 

\section{} Alternatively, you may have a more technical subject such as a design change that must be made, or a study that you wish to direct.  

\subsection{}\label{test2} Subtopics are often broken out, outline style. This is subparagraph~\ref{test2}.

\subsection{} If you have an (a) you have to have a (b); otherwise just collapse it into one subsection.

\section{NAVSEA 05H Request} As another example, NAVSEA 05H recommends a series of tow tank tests to address issues with the hydrodynamics of a particular ship.

\section{NAVSEA 05H Discussion} During sea trials of a different ship, an issue was discovered with a particular operation. Cross project review of all ships shows that the same issue could occur in this particular ship.  In reference (d), NSWC Carderock has been tasked with tow tank and scale model studies to examine this issue.  NAVSEA 08 support of these studies is needed to ensure proper assumptions.

\section{} In a real situation you would actually spell out the technical details required to understand the problem and decision here.  Space is at a premium - you want to describe enough detail to understand the problem, however, you do not want to distract the decision makers with "noise" such as irrelevant details, or less important information.  It is always a tough balance to strike. 

\section{NAVSEA 08 Action and Request for Bettis Action}  NAVSEA 05H request is approved. Within 2 weeks of receipt of this letter, Bettis is directed to provide X Y and Z in support of the evaluations. In technical work it helps to be crystal clear as to what is recommended or directed. Resist the temptation to hedge bets or be like a double-talking politician; technology is difficult enough as is so it is incumbent upon engineers to be clear and concise; to have technical integrity; and to be decisive in recommendations.  Dissent is also allowed and must be documented when a decision is made. 

\section{Discussion} A super wamodyne drive system (SWDS), in which water is manipulated around the hull via a system of force fields emitted by graviton and anti-graviton emitters, may have the following potential benefits:

\subsection{} Arrangements not limited by the presence of a central shaft.

\subsection{} Efficiency improvements by fine control of the boundary layer and propulsor-hull interactions.

\subsection{} Stealth improvement. 

\subsection{} Maintenance and lifecycle cost improvements enabled by the lack of hull penetrations.

\section{NAVSEA Discussion} The extent to which these benefits can be realized in an actual ship design are currently unknown.  In particular, the following are major known risks of this technology that could make SWDS a non-starter in a marine environment:

\subsection{} Wamodyne fields are known to cause seawater to explode

\subsection{} The complete lack of a program for graviton manipulation, materials characterization, system design, etc.

\subsection{} Adverse impacts to operator health from prolonged exposure to anti-gravitons.

\subsection{} Growth in system weight and volume requirements when scaling from single-use test rigs to designs intended for dedicated long term service. 

\section{Request for Contractor Action} To identify if these benefits are realistic, Electric Boat (teamed with KAPL) and NGNN (teamed with Bettis) are requested to generate notional propulsion designs incorporating this technology, in concert with ship studies directed by NAVSEA letter Ser 05U/16-1234.  

\section{} The action taken by this letter is considered by the Government to be within to scope of existing contract N-12345678. 

%\respectfully{}
\noclosing{}
\signspace{}
\signature{I P Freely}
%\sendertitle{ADM, USN}
\bydirection

%\copytonextpage
%\newpage
\copyto
General Manager, KAPL\\
Manager, ARP, KAPL\\
Manager, MER, Bettis\\
Director, Nuclear Engineering, NNSD\\
Vice President, Nuclear Engineering, GDEB\\

% record note prototype
\navyrecordnote

\navyrecordnotedistribution{%
ADM (p)\\
Rodgers (g)\\
Kearney (w)}%
\navyrecordnoteconcurrences{%
\navyrecordnoteconcurrence{08K}
\navyrecordnoteconcurrence{08I}}

\navyrecordnotesubjline

\section{}
The action of this letter is based on a review by Evangelista and comments from Foo and Bar which were resolved or incorporated.

\section{}
Changes happy to glad. 


\end{document}